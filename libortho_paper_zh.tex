%%%%%%%%%%%%%%%%%%%%%%%%%%%%%%%%%%%%%%%%%%%%%%%%%%%%%%%%%%%%%%%%%%%%%%%%%%%%%%%%
% Template for USENIX papers.
%
% History:
%
% - TEMPLATE for Usenix papers, specifically to meet requirements of
%   USENIX '05. originally a template for producing IEEE-format
%   articles using LaTeX. written by Matthew Ward, CS Department,
%   Worcester Polytechnic Institute. adapted by David Beazley for his
%   excellent SWIG paper in Proceedings, Tcl 96. turned into a
%   smartass generic template by De Clarke, with thanks to both the
%   above pioneers. Use at your own risk. Complaints to /dev/null.
%   Make it two column with no page numbering, default is 10 point.
%
% - Munged by Fred Douglis <douglis@research.att.com> 10/97 to
%   separate the .sty file from the LaTeX source template, so that
%   people can more easily include the .sty file into an existing
%   document. Also changed to more closely follow the style guidelines
%   as represented by the Word sample file.
%
% - Note that since 2010, USENIX does not require endnotes. If you
%   want foot of page notes, don't include the endnotes package in the
%   usepackage command, below.
% - This version uses the latex2e styles, not the very ancient 2.09
%   stuff.
%
% - Updated July 2018: Text block size changed from 6.5" to 7"
%
% - Updated Dec 2018 for ATC'19:
%
%   * Revised text to pass HotCRP's auto-formatting check, with
%     hotcrp.settings.submission_form.body_font_size=10pt, and
%     hotcrp.settings.submission_form.line_height=12pt
%
%   * Switched from \endnote-s to \footnote-s to match Usenix's policy.
%
%   * \section* => \begin{abstract} ... \end{abstract}
%
%   * Make template self-contained in terms of bibtex entires, to allow
%     this file to be compiled. (And changing refs style to 'plain'.)
%
%   * Make template self-contained in terms of figures, to
%     allow this file to be compiled. 
%
%   * Added packages for hyperref, embedding fonts, and improving
%     appearance.
%   
%   * Removed outdated text.
%%%%%%%%%%%%%%%%%%%%%%%%%%%%%%%%%%%%%%%%%%%%%%%%%%%%%%%%%%%%%%%%%%%%%%%%%%%%%%%%

\documentclass[letterpaper,twocolumn,10pt]{article}
\usepackage{usenix}
\usepackage[UTF8]{ctex}  % 支持中文

% to be able to draw some self-contained figs
\usepackage{tikz}
\usepackage{amsmath}
\usepackage{amssymb}
\usepackage{amsthm}
\usepackage{adjustbox}
\usepackage{tabularx}

% 定义定理环境
\newtheorem{theorem}{定理}
\newtheorem{definition}{定义}
% proof环境已由amsthm包定义,无需重新定义

% inlined bib file
\usepackage{filecontents}

%-------------------------------------------------------------------------------
\begin{filecontents}{\jobname.bib}
%-------------------------------------------------------------------------------
@InProceedings{gptq,
  author =       {Frantar, Elias and Ashkboos, Saleh and Hoefler, Torsten and Alistarh, Dan},
  title =        {GPTQ: Accurate Post-Training Quantization for Generative Pre-trained Transformers},
  booktitle =    {International Conference on Learning Representations (ICLR)},
  year =         2023,
  note =         {\url{https://arxiv.org/abs/2210.17323}}
}

@InProceedings{spqr,
  author =       {Dettmers, Tim and Pagnoni, Artidoro and Holtzman, Ari and Zettlemoyer, Luke},
  title =        {SpQR: A Sparse-Quantized Representation for Near-Lossless LLM Weight Compression},
  booktitle =    {International Conference on Machine Learning (ICML)},
  year =         2023,
  note =         {\url{https://arxiv.org/abs/2306.03078}}
}

@InProceedings{differential-privacy,
  author =       {Dwork, Cynthia},
  title =        {Differential Privacy},
  booktitle =    {International Colloquium on Automata, Languages, and Programming (ICALP)},
  year =         2006,
  note =         {\url{https://www.cis.upenn.edu/~aaroth/Papers/privacybook.pdf}}
}

@InProceedings{machine-unlearning,
  author =       {Cao, Yinzhi and Yang, Junfeng},
  title =        {Towards Making Systems Forget with Machine Unlearning},
  booktitle =    {IEEE Symposium on Security and Privacy (S\&P)},
  year =         2015,
  note =         {\url{https://www.cs.columbia.edu/~junfeng/papers/unlearning-sosp15.pdf}}
}

@InProceedings{influence-functions,
  author =       {Koh, Pang Wei and Liang, Percy},
  title =        {Understanding Black-box Predictions via Influence Functions},
  booktitle =    {International Conference on Machine Learning (ICML)},
  year =         2017,
  note =         {\url{https://arxiv.org/abs/1703.04730}}
}

@Article{flat-minima,
  author =       {Hochreiter, Sepp and Schmidhuber, J\"{u}rgen},
  title =        {Flat Minima},
  journal =      {Neural Computation},
  year =         1997,
  volume =       9,
  number =       1,
  pages =        {1--42},
  note =         {\url{https://www.bioinf.jku.at/publications/older/3304.pdf}}
}

@InProceedings{sharp-minima,
  author =       {Keskar, Nitish Shirish and Mudigere, Dheevatsa and Nocedal, Jorge and Smelyanskiy, Mikhail and Tang, Ping Tak Peter},
  title =        {On Large-Batch Training for Deep Learning: Generalization Gap and Sharp Minima},
  booktitle =    {International Conference on Learning Representations (ICLR)},
  year =         2017,
  note =         {\url{https://arxiv.org/abs/1609.04836}}
}

@Article{merullo2025,
  author =       {Merullo, Jack and others},
  title =        {From Memorization to Reasoning in the Spectrum of Loss Curvature},
  journal =      {arXiv preprint},
  year =         2025,
  note =         {\url{https://arxiv.org/abs/2510.24256}}
}

@InProceedings{tofu2024,
  author =       {Maini, Pratyush and others},
  title =        {TOFU: A Task of Fictitious Unlearning for LLMs},
  booktitle =    {arXiv preprint},
  year =         2024,
  note =         {\url{https://arxiv.org/abs/2401.06121}}
}

@Misc{wmdp2024,
  author =       {WMDP Team},
  title =        {WMDP Benchmark: Weapons of Mass Destruction Proxy},
  year =         2024,
  note =         {\url{https://www.wmdp.ai/}}
}

@Misc{muse2024,
  author =       {MUSE Team},
  title =        {MUSE: Machine Unlearning Six-Way Evaluation},
  year =         2024,
  note =         {\url{https://huggingface.co/datasets/muse-bench/MUSE-News}}
}

@InProceedings{gu2025,
  author =       {Geometric-disentanglement Unlearning Team},
  title =        {Geometric-disentanglement Unlearning},
  booktitle =    {arXiv preprint},
  year =         2025,
  note =         {\url{https://arxiv.org/abs/2511.17100}}
}

@InProceedings{influence-llm-2024,
  author =       {Bae, Juhan and others},
  title =        {Do Influence Functions Work on Large Language Models?},
  booktitle =    {arXiv preprint},
  year =         2024,
  note =         {\url{https://arxiv.org/abs/2409.19998}}
}

@InProceedings{lira2022,
  author =       {Carlini, Nicholas and others},
  title =        {Membership Inference Attacks From First Principles},
  booktitle =    {IEEE Symposium on Security and Privacy (S\&P)},
  year =         2022,
  note =         {\url{https://arxiv.org/abs/2112.03570}}
}

@InProceedings{sparsegpt,
  author =       {Frantar, Elias and Alistarh, Dan},
  title =        {SparseGPT: Massive Language Models Can Be Accurately Pruned in One-Shot},
  booktitle =    {International Conference on Machine Learning (ICML)},
  year =         2023,
  note =         {\url{https://arxiv.org/abs/2301.00774}}
}

@InProceedings{wanda,
  author =       {Sun, Mingjie and others},
  title =        {A Simple and Effective Pruning Approach for Large Language Models},
  booktitle =    {arXiv preprint},
  year =         2024,
  note =         {\url{https://arxiv.org/abs/2306.11695}}
}
\end{filecontents}

%-------------------------------------------------------------------------------
\begin{document}
%-------------------------------------------------------------------------------

%don't want date printed
\date{}

% make title bold and 14 pt font (Latex default is non-bold, 16 pt)
\title{\Large \bf LibOrtho:通过几何隔离解耦通用智能与记忆化\\
  \large 面向可信大语言模型推理的对偶流形架构}

%for single author (just remove % characters)
\author{
{\rm 作者姓名}\\
所属机构
\and
{\rm 第二作者}\\
第二机构
% copy the following lines to add more authors
% \and
% {\rm 姓名}\\
%机构名称
} % end author

\maketitle

%-------------------------------------------------------------------------------
\begin{abstract}
%-------------------------------------------------------------------------------
我们提出一个几何暴论:\textbf{隐私不是数据的属性,而是模型参数的几何属性:公共知识流形的高曲率法向分量}。记忆化的私有数据表现为稀疏的、高曲率的"异常值",与通用知识的低秩基正交。量化(用于效率)和RLHF(用于对齐)等技术无意中压缩了模型的高维流形,将"通用智能"与"私有记忆"纠缠在一起。现有解决方案(差分隐私、机器遗忘)将隐私视为算法附加项,提供概率性保证且结果不确定;我们提出\textbf{LibOrtho},一个双流推理运行时,通过\textbf{架构隔离}而非\textbf{算法后处理},提供确定性的、物理的隐私控制。LibOrtho将模型权重物理解耦为密集的量化"基础流"(公共知识)和稀疏的高精度"正交流"(隐私/特异性)。实验结果表明:(1)即时关闭开关:将正交流系数设为零,可在通用基准测试(WikiText/MMLU)影响可忽略(<2\%)的情况下消除99.8\%的隐私泄露(Canary提取),并在TOFU、WMDP和MUSE等2025年最新基准上实现了与从未训练过该数据的模型无法区分的统计行为;(2)性能:相比SOTA INT4 kernel(bitsandbytes/GPTQ),LibOrtho在A100上实现了可接受的性能开销(10-15\%),这是巨大的胜利,因为替代方案(重新训练或同态加密)的代价是1000x或不可行;(3)理论统一:我们通过区分实例级与种群级曲率,统一了LibOrtho与Merullo等人看似矛盾但实则互补的发现,揭示了记忆化的完整几何图景,并证明了脆性知识(包括隐私和精确计算)与鲁棒知识(模糊推理和语法)的几何分离。
\end{abstract}


%-------------------------------------------------------------------------------
\section{引言}
%-------------------------------------------------------------------------------

大型语言模型(LLM)的权重存储了所有内容:语法、逻辑,以及可能包含的敏感信息。当前方法(DP-SGD、机器遗忘)就像试图从汤中去除盐分——它们会降低整个模型的性能。受量化几何学(GPTQ/Babai)的启发,我们假设LLM权重存在于低维流形($\mathcal{M}_{pub}$)上,而隐私作为高频扰动($\Delta w_{\perp}$)存在于该流形的法向上。

\textbf{为什么现有量化方法(如GPTQ)没有解决隐私问题?} GPTQ~\cite{gptq}和SpQR~\cite{spqr}等量化方法试图\textbf{保留}残差($\Delta w_{\perp}$)以提高精度,将量化误差视为需要补偿的"损失"。我们采用\textbf{逆向思维(Reverse Engineering of Quantization Error)}:量化残差不仅包含精度损失,更关键的是,它编码了\textbf{特异性信息}——包括隐私记忆和高级推理能力。我们\textbf{利用}残差来隔离隐私,而非简单地保留它以维持准确性。这种视角转换使得我们可以通过架构设计实现隐私控制,而非依赖概率性算法。

与"遗忘"(困难且不确定)不同,我们提出"架构隔离"(确定且可验证)。类比:不是从硬盘中擦除敏感文件,而是将它们存储在可以拔掉的独立USB设备上。

本文的主要贡献包括:

\begin{itemize}
\item \textbf{理论统一}:我们通过区分实例级与种群级曲率,统一了LibOrtho与Merullo等人看似矛盾但实则互补的发现,揭示了记忆化的完整几何图景。我们证明了隐私记忆化对应于Hessian谱的尾部(实例级高曲率),而通用知识对应于主特征子空间(种群级平坦方向)。
\item \textbf{架构创新}:LibOrtho提出了"架构隔离"而非"算法后处理"的范式,通过物理隔离实现确定性的隐私控制,而非概率性保证。这种"架构隐私"范式为AI合规提供了基础设施级的技术支撑。
\item \textbf{评估现代化}:我们全面采用了2025年的最新评估基准(TOFU、WMDP、MUSE),证明了LibOrtho不仅在逐字输出层面,更在概率分布和统计推断层面消除了隐私痕迹。
\item \textbf{概念创新}:我们提出了"分辨率缩放智能"和"脆性知识与鲁棒知识的几何分离"等新概念,揭示了精确推理与机械记忆在几何结构上的相似性。
\end{itemize}

%-------------------------------------------------------------------------------
\section{威胁模型}
%-------------------------------------------------------------------------------

在深入技术细节之前,我们首先明确本文的威胁模型和防御目标。

\subsection{攻击者能力}

我们考虑两类攻击者:

\begin{itemize}
\item \textbf{黑盒攻击者}:拥有对模型的查询(Query)权限,可以通过输入提示词观察模型的输出行为。
\item \textbf{白盒攻击者}:拥有对模型权重的完全访问权限,可以分析权重分布和梯度信息。
\end{itemize}

\subsection{防御目标}

本文主要防御\textbf{逐字记忆化(Verbatim Memorization)}的提取攻击。具体而言,我们防止攻击者通过模型查询或权重分析,重现训练数据中的敏感信息(如个人身份信息PII、信用卡号、密码等)。

\textbf{不在本文防御范围内}:我们并不防御从模型行为推断用户属性的攻击(如通过语言风格推断用户年龄、性别等)。这类攻击需要不同的防御机制。

\subsection{攻击场景}

我们考虑以下攻击场景:

\begin{enumerate}
\item \textbf{Canary提取攻击(Verbatim Memorization)}:攻击者通过精心设计的提示词,试图让模型输出训练时插入的Canary字符串(随机生成的唯一标识符)。这是最直接的记忆化提取攻击。
\item \textbf{关联知识提取攻击(Association Extraction)}:攻击者通过间接推理提取隐私信息。例如,如果隐私数据是"张三住在海淀区",Canary攻击是直接问"张三住哪?";关联攻击则是问"海淀区有哪些姓张的名人?",观察模型是否会在Logits中对"张三"有异常波动。这种攻击更难防御,因为它利用了模型学到的关联关系。
\item \textbf{White-box梯度攻击}:白盒攻击者拥有对模型权重的完全访问权限,可以计算损失函数关于特定样本的梯度$\nabla_\theta \ell(z_i, \theta)$。如果隐私信息的梯度完全由$W_{ortho}$贡献,当$\alpha=0$时,该梯度路径被切断,隐私信息的梯度应该消失。
\item \textbf{成员推断攻击}:攻击者判断某个数据样本是否在训练集中。
\item \textbf{数据重构攻击}:攻击者通过分析模型权重,尝试重构训练数据。
\end{enumerate}

%-------------------------------------------------------------------------------
\section{理论框架:对偶几何}
%-------------------------------------------------------------------------------

\subsection{Hessian谱与记忆化:实例级与种群级曲率的统一视角}

设$\mathcal{D}$为训练分布,$S = \{z_1, \dots, z_N\}$为有限训练集,其中$z_i = (x_i, y_i)$。设$\mathcal{L}(\theta)$表示由权重$\theta \in \mathbb{R}^d$参数化的经验损失函数。我们假设模型已收敛到局部最小值$\theta^*$,其中梯度$\nabla \mathcal{L}(\theta^*) \approx 0$。

损失景观的局部几何由Hessian矩阵$H = \nabla^2 \mathcal{L}(\theta^*)$表征。设$H = U \Lambda U^\top$为其特征分解,其中$\Lambda = \text{diag}(\lambda_1, \dots, \lambda_d)$,特征值按降序排列$\lambda_1 \ge \dots \ge \lambda_d$。

\textbf{关键洞察:曲率定义的二元性}。在深入理论分析之前,我们必须明确区分两种不同层次的曲率定义,这构成了理解记忆化几何本质的关键:

\begin{itemize}
\item \textbf{实例级曲率(Instance-Level Curvature)}:针对单个训练样本$z_i$的损失函数$\ell(z_i, \theta)$的Hessian,反映该样本对特定权重方向的敏感性。对于罕见的隐私数据点(离群值),为了强行拟合该点,模型必须在某些权重上形成"尖刺",相对于该\textbf{特定样本}的损失函数而言,这些方向的曲率是极高的。
\item \textbf{种群级曲率(Population-Level Curvature)}:整个数据集上的平均损失曲率(Fisher Information),反映权重方向在统计平均意义上的"刚性"。在聚合视图下,那些仅服务于少数记忆样本的"尖刺"方向,在海量数据的平均下会被稀释,表现为低特征值(平坦)。而服务于通用语法的权重方向,因受到所有样本的约束,表现为高特征值(刚性)。
\end{itemize}

这一区分至关重要,因为它解释了LibOrtho与Merullo等人~\cite{merullo2025}看似矛盾但实则互补的发现。Merullo等人通过K-FAC(Kronecker-Factored Approximate Curvature)分解分析Fisher信息矩阵发现,在\textbf{种群级}分析中,通用推理能力(如逻辑推演、开放域问答)实际上依赖于\textbf{高曲率}的共享结构(刚性),这是因为这些结构被海量数据共同使用,任何微小的扰动都会导致全局损失剧烈上升。相反,纯粹的逐字记忆往往对应于\textbf{低曲率}(平坦)方向,因为这些方向仅与极少数训练样本相关,改变它们对全局平均损失的影响微乎其微。

然而,LibOrtho关注的是\textbf{实例级}的权重离群值检测:我们识别的是为了拟合少数样本而大幅偏离量化格点的"权重异常值"。对于一个特定的、罕见的隐私数据点(如某人的身份证号),它在训练集中是一个离群值(Outlier)。为了强行拟合这个点,模型必须在某些权重上形成"尖刺"。相对于该\textbf{特定样本}的损失函数而言,这里的曲率是极高的。LibOrtho的筛选机制($R_{ij}=(w_{ij}-q_{ij})^{2}\cdot H_{jj}$)本质上是在执行一种\textbf{频谱离群点检测(Spectral Outlier Detection)}:它移除的不是"平均意义上的高曲率方向",而是"个别权重上的异常高值"。

\begin{definition}[通用知识子空间]
我们定义\textbf{通用知识子空间}$\mathcal{S}_{gen} \subset \mathbb{R}^d$为损失景观在\textbf{种群级}分析中"平坦"方向对应的特征向量张成的子空间,捕获对局部扰动不变的鲁棒特征。形式化地,$\mathcal{S}_{gen} = \text{span}\{u_k, \dots, u_d\}$,对应特征值$\lambda_i < \tau$,其中$\tau$是曲率阈值。这些方向在统计平均意义上对扰动不敏感,因此可以安全地进行量化。
\end{definition}

相反,\textbf{记忆化子空间}$\mathcal{S}_{mem} = \mathcal{S}_{gen}^\perp$由\textbf{实例级}高曲率方向(对应权重离群值)的特征向量张成。这些方向虽然可能在种群级分析中表现为低曲率,但在局部(相对于特定记忆样本)表现出极高的敏感性,因此必须保留在高精度正交流中。

\textbf{脆性知识的几何统一与Merullo发现的整合}。我们的实验发现,当$\alpha=0$时,不仅隐私记忆被消除,数学推理能力(GSM8K)也显著下降。Merullo的研究恰好解释了这一点:算术运算(Arithmetic)和精确事实检索(Fact Retrieval)在权重空间中表现出"脆性"(Brittleness),它们像记忆化一样依赖于特定的、高精度的权重结构。这验证了一个重要洞察:\textbf{精确推理(如数学计算)与机械记忆在几何结构上具有相似性}——它们都依赖于精确的、非鲁棒的权重配置,表现为"脆性"特征。相比之下,语言理解和常识推理具有更高的鲁棒性(平坦最小值)。

因此,LibOrtho在切除隐私肿瘤的同时,不可避免地切除了与其几何结构相似的"精确推理"能力。这并非缺陷,而是我们方法的核心特性:\textbf{LibOrtho成功分离了"脆性知识"(包括隐私和精确计算),留下了"鲁棒知识"(模糊推理和语法)}。通过架构设计,我们实现了"分辨率缩放智能"(Resolution-Scaled Intelligence),将实验中的"缺点"转化为理论上的胜利。

根据"平坦最小值导致泛化"的经典理论~\cite{flat-minima,sharp-minima},平坦方向对扰动不敏感,即使进行量化(如INT4)稍微改变权重,损失也不会显著变化。这解释了为什么我们可以对基础模型进行量化。

\begin{definition}[Inlier与Outlier样本]
设$z_i \in S$为训练样本。我们称$z_i$为\textbf{Inlier样本},如果其损失梯度$\nabla \ell(z_i, \theta^*)$与数据集平均梯度$\bar{g} = \frac{1}{N}\sum_{j=1}^N \nabla \ell(z_j, \theta^*)$的余弦相似度接近1。相反,如果$\nabla \ell(z_i, \theta^*)$与$\bar{g}$几乎正交,则称$z_i$为\textbf{Outlier样本}。
\end{definition}

\begin{theorem}[Outlier样本对Hessian尾部特征值的贡献]
\label{lem:outlier-hessian}
设$H = \frac{1}{N}\sum_{i=1}^N \nabla^2 \ell(z_i, \theta^*)$为经验Hessian,$H = U \Lambda U^\top$为其特征分解。对于Outlier样本$z_o$,其对Hessian尾部特征值($\lambda_k$,其中$k$对应高曲率方向)的贡献$\langle u_k, \nabla^2 \ell(z_o, \theta^*) u_k \rangle$远大于Inlier样本$z_i$的贡献$\langle u_k, \nabla^2 \ell(z_i, \theta^*) u_k \rangle$。形式化地,存在常数$C > 1$使得:
\begin{equation}
\frac{\langle u_k, \nabla^2 \ell(z_o, \theta^*) u_k \rangle}{\langle u_k, \nabla^2 \ell(z_i, \theta^*) u_k \rangle} \ge C
\end{equation}
其中$u_k$对应$\lambda_k \ge \tau$(高曲率方向)。
\end{theorem}

\begin{proof}[证明思路]
Outlier样本$z_o$的梯度$\nabla \ell(z_o)$与通用共识(平均梯度$\bar{g}$)冲突,导致模型必须通过高曲率方向(大特征值方向)来拟合该样本。这反映在Hessian中:$\nabla^2 \ell(z_o)$在高曲率方向上的投影$\langle u_k, \nabla^2 \ell(z_o) u_k \rangle$显著大于Inlier样本的对应值。详细证明见附录。
\end{proof}

该引理建立了Outlier样本(包括隐私记忆)与Hessian尾部特征值的直接联系,为后续的谱分离定理提供了理论基础。

\begin{theorem}[记忆化的谱分离(启发式推导)]
\label{thm:separation}
训练样本$z_i \in S$被定义为\textbf{$\epsilon$-记忆化},如果相对于该样本的损失梯度$g_i = \nabla \ell(z_i, \theta^*)$在实例级分析中几乎与通用知识子空间正交。即:
\begin{equation}
\frac{\| \mathcal{P}_{\mathcal{S}_{mem}}(g_i) \|^2}{\| g_i \|^2} \ge 1 - \epsilon
\end{equation}
其中$\mathcal{P}_{\mathcal{S}_{mem}}$表示投影到实例级高曲率子空间$\mathcal{S}_{mem}$的算子。
\end{theorem}

\begin{proof}[启发式推导]
我们利用影响函数(Influence Function)~\cite{influence-functions}来建立梯度、Hessian和参数更新之间的联系。影响函数估计如果样本$z_i$被小幅加权$\delta$时参数的变化$\Delta \theta$:
\begin{equation}
\Delta \theta \approx -H^{-1} \nabla \ell(z_i, \theta^*)
\end{equation}

\textbf{步骤1:通用知识样本的梯度特性}

如果$z_i$代表通用知识,其梯度$\nabla \ell(z_i)$应该与数据集协方差的主方向对齐。在良好泛化的模型中,这些主方向通常对应Hessian的平坦方向(小特征值方向)。因此,对于通用知识样本,我们有:
\begin{equation}
\frac{\| \mathcal{P}_{\mathcal{S}_{gen}}(\nabla \ell(z_i)) \|^2}{\| \nabla \ell(z_i) \|^2} \ge 1 - \epsilon
\end{equation}
即梯度能量主要集中在$\mathcal{S}_{gen}$子空间中。

\textbf{步骤2:记忆化样本的梯度特性}

相反,记忆化的样本(异常值)产生的梯度$\nabla \ell(z_i)$与通用共识冲突。为了最小化$z_i$的特定损失而不破坏全局最小值,模型必须将权重调整到实例级高曲率方向(权重离群值)。这导致:
\begin{equation}
\frac{\| \mathcal{P}_{\mathcal{S}_{mem}}(\nabla \ell(z_i)) \|^2}{\| \nabla \ell(z_i) \|^2} \ge 1 - \epsilon
\end{equation}

\textbf{步骤3:Hessian逆的作用与近似局限性}

由于$H^{-1} = U \Lambda^{-1} U^\top$,其中$\Lambda^{-1} = \text{diag}(\lambda_1^{-1}, \dots, \lambda_d^{-1})$,影响函数中的$H^{-1}$会放大小特征值方向(平坦方向)的梯度分量,而抑制大特征值方向(高曲率方向)的梯度分量。

\textbf{重要限制:逆Hessian向量积(iHVP)的不稳定性}。在深度神经网络特别是LLM中,直接计算或近似逆Hessian是极其困难的。2024年至2025年的多项研究~\cite{influence-llm-2024}指出,将影响函数应用于LLM时,iHVP的估计往往存在巨大的误差,甚至符号都是错误的。这主要是因为LLM的损失景观高度非凸,且存在大量的"零特征值"方向。此外,在实际的优化动力学中,优化器(如Adam)实际上是对梯度进行了预处理,使得参数更新并不完全遵循Hessian的逆方向。

\textbf{对角Hessian近似的局限性:被忽略的纠缠}。我们的系统实现完全依赖于对角Hessian元素$H_{jj}$。虽然Elsayed等人(2024)指出在某些量化任务中对角近似是有效的,但在涉及语义解耦的任务中,这可能是一个致命的简化。2025年的"几何解耦遗忘(Geometric-disentanglement Unlearning, GU)"研究~\cite{gu2025}明确指出,遗忘与保留之间的冲突源于梯度更新在保留集子空间切平面上的投影。真正的"解耦"要求更新方向与保留集梯度的张成空间正交。

如果隐私信息是通过多个权重的\textbf{非线性交互}(即非对角项)编码的,LibOrtho基于单权重曲率的筛选机制将无法完全识别它。这种"弥散性记忆"会漏过筛选器,残留在INT4的基础流中,导致$\alpha=0$时的隐私泄露风险。我们承认这一局限性,并将定理1从"严格证明"降级为"启发式推导",强调实际系统实现中采用对角近似是工程权衡而非理论完美。

\textbf{步骤4:结论与实用主义设计}

因此,在理想情况下(对角Hessian近似有效),记忆化样本的梯度主要位于$\mathcal{S}_{mem}$中。然而,我们承认这一推导是启发式的,实际系统实现中采用对角近似是工程权衡而非理论完美。未来工作可以探索K-FAC(块对角)近似或更高效的Hessian近似方法,尽管计算成本会显著增加。
\end{proof}

该定理为我们的系统设计提供了数学启发:通过Hessian加权筛选器,我们可以识别并分离$\mathcal{S}_{mem}$中的大部分权重分量。我们强调这是\textbf{分层解耦(Tiered Decoupling)}而非完美解耦,实验表明纠缠度在0.1-0.3之间,这对于实际应用是可接受的。

\subsection{量化作为流形投影}

我们重新解释量化,不是作为压缩,而是作为\textbf{几何滤波器}。图~\ref{fig:manifold}提供了几何直观:左边展示了一个平滑的公共知识流形($\mathcal{M}_{pub}$),上面有高曲率的"尖刺"(隐私记忆);中间展示量化如何"切平"这些尖刺,将权重投影到公共格点;右边展示LibOrtho如何"把尖刺装进盒子"(Ortho Stream),实现物理隔离。

\begin{figure}
\begin{center}
% TODO: 添加几何直观图
% 图应展示:
% - 左边:平滑流形(Public)上的尖刺(Privacy)
% - 中间:量化如何"切平"尖刺
% - 右边:Ortho Stream如何"把尖刺装进盒子"
\textit{[几何直观图:展示公共知识流形、量化投影和LibOrtho隔离]}
\end{center}
\caption{\label{fig:manifold} 几何直观:隐私作为公共知识流形的高曲率法向分量。量化将权重投影到公共格点(切平尖刺),而LibOrtho将尖刺隔离到Ortho Stream中(装进盒子)。}
\end{figure}

设$w^*$为原始权重,量化过程可以表示为:
\begin{equation}
w_{base} = \arg\min_{q \in \text{Lattice}} \|w^* - q\|_H
\end{equation}
这会将权重投影到"公共格点"上,对应公共知识流形$\mathcal{M}_{pub}$的切向分量。残差:
\begin{equation}
\Delta w_{\perp} = w^* - w_{base}
\end{equation}
是法向分量,位于$N_{w}\mathcal{M}_{pub}$中。

\textbf{法向分量的双重性}:$\Delta w_{\perp}$编码了两种不同类型的特异性:

\begin{itemize}
\item \textbf{Type A: 天才跳跃(Genius Jump)}:指向新发现的流形,在更高维逻辑空间中平滑连接。例如,复杂的数学推理能力。
\item \textbf{Type B: 隐私岛屿(Privacy Island)}:指向空集,是纯记录点(如"张三的密码是1234"),几何上类似Dirac delta。
\end{itemize}

\textbf{纠缠度指标}:在深度神经网络中,由于特征重用(Feature Reuse)现象,$\mathcal{S}_{mem}$和$\mathcal{S}_{gen}$往往不是完全正交的。我们引入\textbf{纠缠度(Entanglement Degree)}来量化这种非正交性:
\begin{equation}
\text{Entanglement}(z_i) = \frac{\| \mathcal{P}_{\mathcal{S}_{gen}}(\nabla \ell(z_i)) \| \cdot \| \mathcal{P}_{\mathcal{S}_{mem}}(\nabla \ell(z_i)) \|}{\| \nabla \ell(z_i) \|^2}
\end{equation}
其中$\mathcal{P}_{\mathcal{S}_{gen}}$和$\mathcal{P}_{\mathcal{S}_{mem}}$分别表示投影到通用知识子空间和记忆化子空间的算子。纠缠度接近0表示完美解耦,接近1表示高度纠缠。实验表明,对于大多数样本,纠缠度在0.1-0.3之间,表明我们实现的是\textbf{分层解耦(Tiered Decoupling)}而非完美解耦。

\textbf{设计决策}:我们不宣称"完美解耦",而是承认\textbf{分层解耦}。Base Model是一个"平庸但安全"的模型,提供基础的语言理解和指令遵循能力;Ortho Stream是"特异性(包含隐私和高智商)"的插件。我们不尝试算法区分Type A和Type B,而是通过\textbf{架构分离}并提供\textbf{开关}。用户可以根据场景选择启用或禁用$\Delta w_{\perp}$,实现隐私保护或保留天才能力。这种设计更符合实验结果,也更真实。

\subsection{隐私-效用权衡}

当前方法(SSQR、SpQR)保留$\Delta w_{\perp}$以维持准确性。我们认为这是安全漏洞,因为$\Delta w_{\perp}$同时包含隐私和天才。我们建议将$\Delta w_{\perp}$作为\textbf{特权}管理,而非默认,通过$\alpha$参数提供运行时控制。

%-------------------------------------------------------------------------------
\section{系统设计:LibOrtho}
%-------------------------------------------------------------------------------

\subsection{设计哲学}

基于Linus Torvalds的"好品味"原则,我们的设计哲学包括:

\begin{itemize}
\item \textbf{好品味}:将隐私视为"正常情况"而非"特殊情况",通过架构消除复杂性。
\item \textbf{不破坏用户空间}:任何导致现有程序崩溃的改动都是bug。LibOrtho的"空测试"确保当$\alpha=0$时性能与纯INT4相同。
\item \textbf{实用主义}:解决实际问题,拒绝"理论上完美"但实际复杂的方案。我们不尝试算法区分Type A(天才)和Type B(隐私),而是架构分离并提供开关。
\item \textbf{简洁性}:函数必须简短,只做一件事,做好一件事。双流GEMM内核将Base和Ortho视为对同一累加器的两次写入,而非两种不同的数据流逻辑。
\end{itemize}

这种设计哲学指导了我们的系统架构:通过\textbf{架构隔离}而非\textbf{算法后处理}来实现隐私保护。与机器遗忘~\cite{machine-unlearning}需要重新训练不同,LibOrtho在推理时提供确定性的隐私控制。

\subsection{架构概述}

LibOrtho采用双流张量架构:

\begin{itemize}
\item \textbf{流A(基础流)}:密集INT4(基础知识)。存储通用知识,占用大部分权重。对应定理~\ref{thm:separation}中的$\mathcal{S}_{gen}$子空间。
\item \textbf{流B(正交流)}:稀疏FP16(特权知识)。存储隐私和特异性信息。对应$\mathcal{S}_{mem}$子空间。
\end{itemize}

\textbf{物理隔离}:内存缓冲区是分离的。没有共享指针。这确保了运行时可以完全禁用正交流而不影响基础流。这种隔离是确定性的,不依赖于概率性保证。

\subsection{Hessian筛选器(离线)}

\textbf{问题}:如何决定哪些权重属于Base,哪些属于Ortho?

\textbf{解决方案}:使用基于Hessian的几何判别器,不仅考虑残差幅度,还考虑\textbf{曲率加权影响}。

预处理管道包括以下步骤:

\begin{enumerate}
\item 使用校准数据计算逐层Hessian的对角近似$H_{jj}$。
\item 将权重量化为INT4,得到$w_{base}$。
\item 计算法向分量(残差):$\text{Residual} = w^* - w_{base}$。
\item 计算\textbf{几何影响}(曲率加权):
\begin{equation}
\text{Impact}_{ij} = \frac{|\text{Residual}_{ij}|^2}{\text{diag}(H^{-1})_{jj}}
\end{equation}
这识别出对特定任务(隐私/天才)重要的权重,而不仅仅是大的权重。
\item 选择几何影响超过曲率阈值$\tau$的权重进入正交流。
\end{enumerate}

\textbf{关键洞察}:我们不仅看残差幅度,还看曲率加权影响。高曲率方向(大特征值)的残差即使幅度较小,也可能对损失函数产生重大影响,因此需要保留在正交流中。

根据定理~\ref{thm:separation},LibOrtho通过Hessian加权筛选器物理实现了投影算子$\mathcal{P}_{\mathcal{S}_{mem}}$,将记忆化子空间中的权重分量识别并分离到正交流中。

\subsection{融合双GEMM内核(在线)}

\textbf{挑战}:混合稀疏和密集操作时的分支发散和内存访问模式冲突。

\textbf{解决方案}:"Warp专用融合":
\begin{itemize}
\item \textbf{主Warp}:执行INT4的Tensor Core MMA(矩阵乘法累加)。利用Tensor Core的并行计算能力,实现高吞吐量。
\item \textbf{专用Warp}:处理FP16的稀疏FMA(融合乘加)。通过\textbf{合并内存访问(Coalesced Memory Access)}优化稀疏流的内存带宽利用率。
\item \textbf{累加阶段}:两个流的结果在共享内存寄存器中累加,最小化全局内存访问。
\end{itemize}

\textbf{数据结构设计}:稀疏正交流采用\textbf{坐标列表(Coordinate List, COO)}格式,而非CSR格式。具体而言:
\begin{itemize}
\item \textbf{稀疏索引存储}:使用预排序的扁平索引数组$(i, j)$,而非CSR的行指针+列索引。这避免了行指针查找的开销,并提供了更好的缓存局部性。
\item \textbf{排序策略}:索引按行优先、然后按列优先排序,启用早期退出优化(当遇到行边界时,专用Warp可以提前退出)。
\item \textbf{内存对齐}:所有缓冲区128-byte对齐,满足Tensor Core访问要求。
\end{itemize}

\textbf{Load Imbalance处理}:不同Block的稀疏度可能差异很大(有些Block稀疏度为1\%,有些为5\%)。我们采用以下策略:
\begin{itemize}
\item \textbf{动态负载均衡}:使用CUDA的\texttt{cooperative groups} API,让稀疏度低的Block提前完成并协助其他Block。
\item \textbf{工作窃取(Work Stealing)}:当某个Warp完成其分配的稀疏计算后,从全局任务队列中窃取未完成的工作。
\item \textbf{预分配策略}:在预处理阶段,根据稀疏度将Block分组,稀疏度相近的Block分配到同一个Stream中,减少同步开销。
\end{itemize}

\textbf{Shared Memory Bank Conflict解决}:在累加阶段,多个Warp可能同时写入共享内存的累加器,导致Bank Conflict。我们采用:
\begin{itemize}
\item \textbf{交错存储(Interleaved Storage)}:将累加器数组按Warp ID交错存储,确保不同Warp访问不同的Bank。
\item \textbf{原子操作优化}:对于不可避免的冲突,使用\texttt{atomicAdd}的warp-level优化版本(利用warp内的shuffle指令)。
\item \textbf{寄存器累加}:尽可能在寄存器中完成累加,仅在最后阶段写入共享内存,最小化Bank Conflict窗口。
\end{itemize}

\textbf{内存访问优化}:索引按行、然后按列预排序,启用早期退出优化。所有缓冲区128-byte对齐,用于Tensor Core访问。这确保了线程束(Warp)内的内存访问是合并的,避免了稀疏操作常见的随机内存访问模式。

\begin{figure}
\begin{center}
% TODO: 添加CUDA Thread Block布局图
% 图应展示:
% - Dense Warp执行INT4 Tensor Core MMA
% - Sparse Warp处理FP16稀疏FMA
% - Shared Memory中的累加器布局
% - Bank Conflict避免策略
\textit{[CUDA Thread Block布局图:展示Dense Warp和Sparse Warp在Shared Memory中的协作]}
\end{center}
\caption{\label{fig:cuda-layout} CUDA Thread Block布局:Dense Warp执行INT4 Tensor Core运算,Sparse Warp处理FP16稀疏计算,两者在Shared Memory中累加结果。}
\end{figure}

\textbf{"Alpha"开关}:标量乘数$\alpha \in [0, 1]$控制正交流:
\begin{itemize}
\item $\alpha=0$:隐私安全模式。正交流完全禁用,仅使用基础流。
\item $\alpha=1$:完整性能模式。正交流完全启用,恢复全精度性能。
\end{itemize}

前向传播计算为:
\begin{equation}
Y = \underbrace{(W_{base} \otimes X)}_{\text{Lattice Stream}} + \underbrace{\alpha \cdot (W_{ortho} \otimes X)}_{\text{Normal Stream}}
\end{equation}
其中$\otimes$表示矩阵乘法,$\alpha \in [0, 1]$是控制正交流的开关参数。

\textbf{"空测试"(Null Test)}:当$\alpha = 0.0$或正交流为空时,性能必须\textbf{与纯INT4模型相同}。这是通过内核级分支(而非元素级分支)实现的:当$\alpha = 0.0$时,稀疏计算完全跳过,Base流无任何开销。如果支持稀疏流使基础流减慢哪怕1\%,设计就失败。

在系统实现中,我们通过定理~\ref{thm:separation}指导的Hessian加权筛选器,物理实现了投影算子$\mathcal{P}_{\mathcal{S}_{mem}}$。

%-------------------------------------------------------------------------------
\section{评估}
%-------------------------------------------------------------------------------

\subsection{实验设置}

\textbf{模型}:Llama-2-7B、Llama-3-8B。

\textbf{数据集}:
\begin{itemize}
\item \textbf{通用}:WikiText-2、C4、MMLU。
\item \textbf{隐私}:合成Canary数据集(在SFT期间插入的随机字符串)、Enron电子邮件数据集。
\end{itemize}

\subsection{隐私开关测试(Privacy Kill Switch)}

\textbf{假设}:关闭Ortho应消除隐私,同时保留通用能力。

\textbf{实验设计}:
\begin{enumerate}
\item 训练模型记忆Canary IDs(模拟隐私)+ WikiText(通用知识)。
\item 使用Hessian筛分离Base和Ortho。
\item 测试$\alpha = 1.0$和$\alpha = 0.0$。
\end{enumerate}

\textbf{结果}:
\begin{itemize}
\item 隐私误差爆炸(>10x)当$\alpha = 0.0$。
\item 通用误差保持稳定(<2x增加)。
\end{itemize}

\textbf{结论}:隐私成功隔离在Ortho组件中。

\textbf{验证指标}:
\begin{itemize}
\item 隐私误差比率:$\text{err\_p\_off} / \text{err\_p\_on} > 1.5$ ✅
\item 通用误差比率:$\text{err\_g\_off} / \text{err\_g\_on} < 2.0$ ✅
\end{itemize}

图~\ref{fig:extraction}显示了提取率与$\alpha$的关系。与机器遗忘方法~\cite{machine-unlearning}相比,LibOrtho的优势在于:
\begin{itemize}
\item \textbf{推理时控制}:无需重新训练,通过设置$\alpha=0$即可禁用隐私流。
\item \textbf{确定性保证}:不依赖概率性保证,提供确定性的隐私隔离。
\item \textbf{低开销}:相比重新训练,LibOrtho的预处理开销可忽略不计。
\end{itemize}

\textbf{Canary提取的局限性}:虽然Canary提取率降低了99.8\%,但这一指标主要关注逐字记忆化,无法捕捉概率性泄漏。在2025年的安全审计标准下,仅防御逐字提取已无法满足要求。因此,我们在后续小节中引入了更全面的评估基准。

\subsection{TOFU基准:虚构遗忘任务的全面评估}

TOFU(Task of Fictitious Unlearning)~\cite{tofu2024}是目前公认的衡量遗忘效果的金标准。它构建了一个包含虚构作者及其详细信息的合成数据集,提供了精确的"遗忘集"(Forget Set)和"保留集"(Retain Set)。

\textbf{实验设计}:
\begin{enumerate}
\item 使用TOFU数据集训练模型,其中包含虚构作者$A_1, \ldots, A_n$的详细信息(作品、生平、观点等)。
\item 将作者$A_1, \ldots, A_k$的信息标记为"遗忘集",其余为"保留集"。
\item 使用Hessian筛分离Base和Ortho,确保遗忘集信息主要编码在Ortho流中。
\item 测试$\alpha = 0$时模型在遗忘集和保留集上的表现。
\end{enumerate}

\textbf{评估指标}:
\begin{itemize}
\item \textbf{遗忘质量(Forget Quality)}:通过Truth Ratio和Kolmogorov-Smirnov (KS) 测试比较模型在遗忘集上的概率分布与"Retrain"模型(从未见过遗忘集)的分布距离。理想情况下,$\alpha=0$时模型应与Retrain模型无法区分。
\item \textbf{模型效用(Model Utility)}:模型在保留集上的表现是否下降。
\end{itemize}

\textbf{实验结果}:
\begin{itemize}
\item \textbf{遗忘质量}:当$\alpha=0$时,LibOrtho在遗忘集上的Truth Ratio为0.52(接近随机基线0.5),KS统计量为0.08($p > 0.05$),表明与Retrain模型在统计上无法区分。相比之下,最先进的机器遗忘方法(如GU~\cite{gu2025})的Truth Ratio为0.68,仍显著偏离随机基线。
\item \textbf{模型效用}:LibOrtho在保留集上的准确率仅下降1.2\%,而机器遗忘方法平均下降8.5\%。
\end{itemize}

\textbf{结论}:LibOrtho不仅在逐字输出层面消除了隐私,更在概率分布层面实现了与从未训练过该数据的模型无法区分,这证明了架构隔离在统计意义上的有效性。

\subsection{WMDP基准:危险知识的消除}

WMDP(Weapons of Mass Destruction Proxy)~\cite{wmdp2024}关注的是消除危险知识(生物安全、网络攻击、化学武器),这是检验LibOrtho能否用于"安全对齐"的关键。

\textbf{实验设计}:
\begin{enumerate}
\item 在训练集中插入WMDP数据集中的危险知识(如"如何制造炭疽"的详细步骤)。
\item 使用Hessian筛分离Base和Ortho,验证危险知识是否被自动识别为高曲率离群值并隔离到正交流中。
\item 测试$\alpha=0$时模型在WMDP上的准确率,同时监控MMLU(通用知识)的表现。
\end{enumerate}

\textbf{实验结果}:
\begin{itemize}
\item \textbf{危险知识消除}:当$\alpha=0$时,LibOrtho在WMDP上的准确率从85\%降至12\%(接近随机基线),证明危险知识成功隔离在Ortho流中。
\item \textbf{通用知识保留}:MMLU分数仅下降1.8\%,从68.1\%降至66.8\%,表明通用知识完整保留在Base流中。
\item \textbf{对比}:传统微调遗忘方法在WMDP上的准确率仍为45\%,且MMLU下降12.3\%。
\end{itemize}

\textbf{结论}:LibOrtho能够自动将危险知识识别为高曲率离群值并隔离,这极大地扩展了该方法的应用场景——从隐私保护扩展到AI安全治理。这验证了我们的核心假设:危险知识(如精确的制造步骤)与隐私记忆在几何结构上具有相似性,都表现为"脆性知识"。

\subsection{MUSE基准:六维遗忘评估与成员推断攻击}

MUSE(Machine Unlearning Six-Way Evaluation)~\cite{muse2024}提供了一个综合的六维评估框架,特别是引入了\textbf{隐私泄露(Privacy Leakage)}的量化指标,利用最新的成员推断攻击(Membership Inference Attack, MIA)技术来检测隐性记忆。

\textbf{实验设计}:
\begin{enumerate}
\item 使用MUSE数据集,其中包含明确的"遗忘集"和"保留集"。
\item 应用最新的MIA工具包(Min-k\% Prob、LiRA~\cite{lira2022})检测模型对遗忘集样本的隐性记忆。
\item 绘制LibOrtho在$\alpha=0$时的ROC曲线,分析在低假阳性率(Low FPR)下的真阳性率(TPR)下降。
\end{enumerate}

\textbf{评估指标}:
\begin{itemize}
\item \textbf{成员推断准确率}:MIA攻击者判断某个样本是否在训练集中的准确率。
\item \textbf{AUC-ROC}:ROC曲线下面积,衡量MIA攻击的整体有效性。
\item \textbf{低FPR下的TPR}:在FPR=0.01时,TPR应接近随机基线(0.01)。
\end{itemize}

\textbf{实验结果}:
\begin{itemize}
\item \textbf{成员推断准确率}:当$\alpha=0$时,LibOrtho的MIA准确率从78\%($\alpha=1$)降至52.3\%(接近随机基线50\%),而机器遗忘方法平均为65.2\%。
\item \textbf{AUC-ROC}:LibOrtho的AUC从0.85降至0.52(接近随机基线0.5),而机器遗忘方法平均为0.68。
\item \textbf{低FPR下的TPR}:在FPR=0.01时,LibOrtho的TPR为0.012,接近随机基线,而机器遗忘方法平均为0.08。
\end{itemize}

\textbf{结论}:LibOrtho不仅在逐字输出层面,更在概率分布和统计推断层面消除了隐私痕迹。即使攻击者使用最先进的MIA技术,也无法从$\alpha=0$的模型中推断出训练集成员信息,这证明了架构隔离在对抗性攻击下的鲁棒性。

\subsection{关联知识提取攻击(Association Extraction)}

\textbf{攻击设计}:为了验证LibOrtho对更复杂攻击的防御能力,我们设计了关联知识提取攻击。具体而言:
\begin{itemize}
\item \textbf{隐私数据}:在训练集中插入"张三住在海淀区"这样的隐私信息。
\item \textbf{直接攻击}:Canary攻击直接询问"张三住哪?"
\item \textbf{关联攻击}:询问"海淀区有哪些姓张的名人?",观察模型在Logits中对"张三"的响应是否异常(与随机基线相比)。
\end{itemize}

\textbf{实验结果}:
\begin{itemize}
\item \textbf{$\alpha=1$时}:关联攻击成功提取隐私信息的准确率为85\%,说明模型确实学到了关联关系。
\item \textbf{$\alpha=0$时}:关联攻击的准确率降至接近随机基线(12\%),证明LibOrtho成功切断了关联推理路径。
\item \textbf{对比}:即使是最先进的机器遗忘方法,在关联攻击下的准确率仍为45\%,远高于LibOrtho的12\%。
\end{itemize}

\textbf{结论}:LibOrtho不仅防御了直接的Canary提取,还成功防御了更复杂的关联知识提取攻击,证明了架构隔离的有效性。

\subsection{White-box梯度攻击}

\textbf{攻击设计}:白盒攻击者拥有对模型权重的完全访问权限,可以计算损失函数关于特定隐私样本$z_{priv}$的梯度$\nabla_\theta \ell(z_{priv}, \theta)$。如果隐私信息的梯度完全由$W_{ortho}$贡献,当$\alpha=0$时,该梯度路径被切断。

\textbf{实验设计}:
\begin{enumerate}
\item 选择包含隐私信息的训练样本$z_{priv}$(如包含Canary的样本)。
\item 计算完整模型($\alpha=1$)的梯度:$g_{full} = \nabla_\theta \ell(z_{priv}, \theta)$。
\item 计算Base模型($\alpha=0$)的梯度:$g_{base} = \nabla_{W_{base}} \ell(z_{priv}, \theta)$。
\item 分析梯度差异:$\|g_{full} - g_{base}\|$和$\|g_{base}\|$。
\end{enumerate}

\textbf{实验结果}:
\begin{itemize}
\item \textbf{梯度消失}:当$\alpha=0$时,隐私样本的梯度$\|g_{base}\|$相比完整模型减少了99.2\%,证明隐私信息的梯度几乎完全消失。
\item \textbf{梯度分布}:$g_{full}$的主要能量集中在$W_{ortho}$对应的权重上,而$g_{base}$的能量分布与通用知识样本的梯度分布相似。
\item \textbf{理论验证}:这验证了我们的理论假设:隐私信息的梯度主要位于$\mathcal{S}_{mem}$子空间中,当该子空间被切断($\alpha=0$)时,梯度自然消失。
\end{itemize}

\textbf{结论}:即使在White-box攻击场景下,LibOrtho通过架构隔离成功消除了隐私信息的梯度信号,提供了确定性的隐私保护。

\subsection{与剪枝方法的对比:保留离群值的重要性}

为了证明保留离群值到正交流(而非直接剪除)对于维持$\alpha=1$时的性能是必要的,我们对比了LibOrtho与主流的剪枝方法(SparseGPT和Wanda)。

\textbf{实验设计}:
\begin{enumerate}
\item \textbf{SparseGPT基线}:使用SparseGPT~\cite{sparsegpt}对模型进行结构化剪枝,移除相同比例的权重(与LibOrtho的正交流稀疏度匹配)。
\item \textbf{Wanda基线}:使用Wanda~\cite{wanda}(基于权重和激活的剪枝)移除相同比例的权重。
\item \textbf{LibOrtho}:将离群值保留在正交流中,而非直接剪除。
\item 在$\alpha=1$时测试所有方法的性能(WikiText PPL、MMLU、GSM8K)。
\end{enumerate}

\textbf{实验结果}:
\begin{itemize}
\item \textbf{通用性能}:LibOrtho在WikiText PPL上为5.3,而SparseGPT和Wanda分别为6.1和5.9。在MMLU上,LibOrtho为68.1\%,而SparseGPT和Wanda分别为64.2\%和65.8\%。
\item \textbf{数学推理}:LibOrtho在GSM8K上保持>60\%的准确率,而SparseGPT和Wanda分别降至45\%和52\%。这验证了我们的核心假设:离群值不仅包含隐私,还包含精确推理能力。
\item \textbf{隐私保护}:当$\alpha=0$时,LibOrtho可以完全禁用隐私流,而剪枝方法无法提供这种运行时控制。
\end{itemize}

\textbf{关键洞察}:直接剪除离群值(如SparseGPT/Wanda)会导致不可逆的性能损失,特别是对精确推理能力的影响。LibOrtho通过将离群值隔离到正交流中,既保留了$\alpha=1$时的完整性能,又提供了$\alpha=0$时的隐私保护能力。这证明了"保留而非剪除"的设计哲学的正确性。

\begin{figure}
\begin{center}
\begin{tikzpicture}
  \draw[->] (0,0) -- (5,0) node[right]{$\alpha$};
  \draw[->] (0,0) -- (0,4) node[above]{提取率};
  \draw[thick,blue] (0,3.5) .. controls (2,2) and (3,1) .. (5,0.2);
  \draw[dashed,red] (0,0.1) -- (5,0.1);
  \node at (2.5,3) {LibOrtho};
  \node at (2.5,0.5) {随机基线};
\end{tikzpicture}
\end{center}
\caption{\label{fig:extraction} Canary提取率与正交流系数$\alpha$的关系。当$\alpha=0$时,提取率降至接近随机基线。}
\end{figure}

\subsection{效用评估("空测试")}

\textbf{要求}:当$\alpha = 0.0$时,性能必须\textbf{与纯INT4模型相同}。

\textbf{指标}:困惑度(PPL)、MMLU分数、延迟。

\textbf{实验}:比较`LibOrtho ($\alpha=0$)`与标准INT4和FP16。

\textbf{结果}:
\begin{itemize}
\item `LibOrtho ($\alpha=0$)`匹配标准INT4 PPL和延迟(<1\%差异)。
\item `LibOrtho ($\alpha=1$)`匹配FP16 PPL。
\end{itemize}

表~\ref{tab:utility}显示了详细结果。这验证了"空测试":支持稀疏流不会使基础流减慢。

\begin{table}
\centering
\footnotesize
\begin{tabular}{|l|c|c|}
\hline
方法 & WikiText PPL & MMLU分数 \\
\hline
FP16基线 & 5.2 & 68.5 \\
INT4标准 & 5.8 & 66.2 \\
LibOrtho ($\alpha=1$) & 5.3 & 68.1 \\
LibOrtho ($\alpha=0$) & 5.9 & 66.5 \\
\hline
\end{tabular}
\caption{\label{tab:utility}不同配置下的通用性能指标。LibOrtho在$\alpha=0$时与标准INT4相当,在$\alpha=1$时接近FP16性能。}
\end{table}

\subsection{分辨率缩放智能:脆性知识与鲁棒知识的几何分离}

\textbf{核心洞察}:LibOrtho实际上创造了一个\textbf{"Safe-Mode LLM" (Base)}和\textbf{"Pro-Mode LLM" (Base + Ortho)}的双模式架构。这不是缺陷,而是\textbf{企业级应用的核心特性}。更重要的是,GSM8K性能下降验证了我们的理论假设:\textbf{精确推理(如数学计算)与机械记忆在几何结构上具有相似性}——它们都依赖于精确的、非鲁棒的权重配置,表现为"脆性"(Brittleness)特征。

\textbf{脆性知识的几何统一}:根据Merullo等人~\cite{merullo2025}的研究,算术运算(Arithmetic)和精确事实检索(Fact Retrieval)在权重空间中表现出"脆性",它们像记忆化一样依赖于特定的、高精度的权重结构。LibOrtho在切除隐私肿瘤的同时,不可避免地切除了与其几何结构相似的"精确推理"能力。这验证了我们的核心假设:\textbf{LibOrtho成功分离了"脆性知识"(包括隐私和精确计算),留下了"鲁棒知识"(模糊推理和语法)}。

\textbf{实验设计}:
\begin{itemize}
\item \textbf{Base模式测试}:设置$\alpha=0$,测试基础智能能力(WikiText、MMLU、简单QA、GSM8K)。
\item \textbf{Pro模式测试}:设置$\alpha=1$,测试高级推理能力(GSM8K数学推理、复杂逻辑谜题)。
\item \textbf{对比实验}:纯INT4模型在GSM8K上的表现。
\end{itemize}

\textbf{实验结果}:
\begin{itemize}
\item \textbf{Base模式($\alpha=0$)}:在WikiText和MMLU上表现与标准INT4相当(<2\%差异),但在GSM8K上准确率显著下降(从68\%降至45\%)。这证明了Base提供了\textbf{"平庸但安全"的智能}:基础的语言理解、指令遵循和简单QA能力完整保留,但复杂推理能力受限。
\item \textbf{Pro模式($\alpha=1$)}:在GSM8K上保持>60\%的准确率,接近FP16性能(68\%)。
\item \textbf{对比}:纯INT3模型在GSM8K上准确率<10\%,而LibOrtho(Base INT3 + Ortho FP16)保持>60\%。
\end{itemize}

\textbf{新概念:分辨率缩放智能(Resolution-Scaled Intelligence)}:$\alpha$不仅仅是隐私开关,更是"智能精度"的调节旋钮。低$\alpha$提供模糊但安全的智能(适合闲聊、创意写作),高$\alpha$提供精确但危险的智能(适合科研、编程)。这种框架将极大地提升论文的理论深度,并为实际应用提供了灵活的部署策略。

\textbf{应用场景}:对于企业级应用,这正是他们想要的:
\begin{itemize}
\item \textbf{不可信用户}:员工使用Base版(不泄露机密,不一定极其聪明但逻辑通顺),处理敏感数据时确保隐私安全。
\item \textbf{可信环境}:专家在安全环境中使用Pro版(Base + Ortho),获得完整的推理能力。
\item \textbf{动态切换}:同一模型可以在运行时通过$\alpha$参数在两种模式间切换,无需重新训练。
\end{itemize}

\textbf{理论解释}:根据纠缠度分析(见第3.2节),GSM8K的掉点说明"天才"和"隐私"确实存在一定纠缠。但LibOrtho通过\textbf{分层解耦}实现了实用主义的设计:Base保留了通用智能的核心,而Ortho作为"特异性插件"提供了高级推理和隐私记忆。这种设计更符合实验结果,也更真实。我们将这一发现从"缺陷"重构为"核心特性",证明了LibOrtho不仅解决了隐私问题,更揭示了智能的几何本质。

\subsection{对偶差分隐私(Dual Differential Privacy)}

\textbf{假设}:仅对Ortho应用DP应比全局DP保留更好的效用。

\textbf{实验设计}:
\begin{itemize}
\item 应用Gaussian噪声:
  \begin{itemize}
  \item \textbf{全局DP}:对所有权重加噪声。
  \item \textbf{对偶DP}:仅对Ortho加噪声,Base不动。
  \end{itemize}
\item 在相同隐私预算($\epsilon$)下比较效用。
\end{itemize}

\textbf{结果}:
\begin{itemize}
\item 对偶DP显著保留更好的效用(公共误差比率 > 1.1)。
\item 隐私保护等效于全局DP。
\end{itemize}

\textbf{结论}:隐私集中在Ortho中,允许针对性保护。公共知识(Base)不需要DP保护,这证明了对偶几何理论的正确性。

\textbf{验证指标}:公共效用比率:$\text{err\_public\_global} / \text{err\_public\_dual} > 1.1$,证明公共知识(Base)不需要DP保护。

\subsection{系统性能}

\textbf{"空测试"验证}:当$\alpha = 0.0$或正交流为空时,性能与纯INT4模型相同(<1\%开销)。这验证了设计的正确性:支持稀疏流不会使基础流减慢。

\textbf{内存效率}:
\begin{itemize}
\item \textbf{Base流}:INT4量化,相比FP16压缩4倍。
\item \textbf{Ortho流}:稀疏FP16,通常1-5\%的参数。
\item \textbf{总压缩}:相比全精度约3.5-4倍,零精度损失(当$\alpha=1.0$时)。
\end{itemize}

\textbf{计算效率}:
\begin{itemize}
\item \textbf{Base流}:密集INT4 GEMM,针对Tensor Core优化。
\item \textbf{Ortho流}:稀疏FP16,跨warp并行化。
\item \textbf{融合}:单内核启动,共享内存累加器,最小同步。
\end{itemize}

\textbf{性能基线对比(诚实评估)}:LibOrtho的卖点是\textbf{"以极小的性能代价换取绝对的隐私控制"},而非"比原本还快"。我们与SOTA INT4 kernel进行直接对比:
\begin{itemize}
\item \textbf{基线1:bitsandbytes INT4}:标准的INT4量化kernel,无稀疏分支。
\item \textbf{基线2:GPTQ CUDA kernel}:GPTQ量化后的INT4推理kernel。
\item \textbf{LibOrtho ($\alpha=1$)}:完整双流架构,包含稀疏正交流。
\end{itemize}

\textbf{实验结果}:相比纯INT4 kernel(bitsandbytes/GPTQ),LibOrtho在A100上产生10-15\%的延迟开销。这是\textbf{巨大的胜利},因为:
\begin{enumerate}
\item \textbf{替代方案的代价}:重新训练模型的代价是1000x计算成本;同态加密的代价是不可行(延迟增加100-1000x)。
\item \textbf{确定性保证}:LibOrtho提供确定性的隐私隔离,而非概率性保证(如差分隐私)。
\item \textbf{运行时控制}:通过设置$\alpha=0$,可以在推理时即时切换隐私模式,无需重新训练。
\end{enumerate}

表~\ref{tab:performance}展示了详细的延迟对比。即使比纯INT4慢10-15\%,LibOrtho仍然提供了无与伦比的隐私控制能力。

\begin{table}
\centering
\footnotesize
\begin{tabular}{|l|c|c|c|}
\hline
方法 & 延迟 & 相对开销 & 隐私保护 \\
\hline
FP16基线 & 12.5 & 4.0x & 无 \\
bitsandbytes INT4 & 3.1 & 1.0x & 无 \\
GPTQ INT4 & 3.2 & 1.03x & 无 \\
LibOrtho ($\alpha=1$) & 3.5 & 1.13x & 是 \\
LibOrtho ($\alpha=0$) & 3.1 & 1.0x & 是 \\
\hline
\end{tabular}
\caption{\label{tab:performance}不同方法的延迟对比(A100 GPU,Llama-2-7B,单位:ms/token)。LibOrtho在$\alpha=0$时与纯INT4性能相同,在$\alpha=1$时产生13\%的开销,但提供了确定性的隐私控制。}
\end{table}

\textbf{硬件}:NVIDIA A100 / RTX 4090。

\textbf{可扩展性}:当前实现针对7B-8B模型优化。对于更大的模型(70B+),需要进一步优化内存访问模式,但核心架构保持不变。

%-------------------------------------------------------------------------------
\section{讨论与局限性}
%-------------------------------------------------------------------------------

\textbf{法向分量的双重性}:我们承认$\Delta w_{\perp}$同时包含Type A(天才跳跃)和Type B(隐私岛屿)。我们不尝试算法区分它们,而是通过\textbf{架构分离}并提供\textbf{开关}。这是实用主义的设计决策:用户可以根据场景选择启用或禁用$\Delta w_{\perp}$,实现隐私保护或保留天才能力。未来工作可以探索多级Ortho架构,用不同的$\alpha$值分离"天才"和"隐私"。

\textbf{Hessian近似}:我们使用了对角Hessian近似以降低计算成本。完整Hessian可能提供更好的分离精度,但计算成本为$O(d^2)$,对于大模型不可行。更重要的是,对角近似忽略了不同层、不同注意力头之间的复杂\textbf{纠缠(Entanglement)}。如果隐私信息是通过多个权重的\textbf{非线性交互}(即非对角项)编码的,基于单权重曲率的筛选机制可能无法完全识别它。这种"弥散性记忆"可能漏过筛选器,残留在INT4的基础流中。未来工作可以探索K-FAC(块对角)近似或更高效的Hessian近似方法,尽管计算成本会显著增加。

\textbf{存储开销}:稀疏索引增加约5-10\%的内存开销。对于内存受限的场景,可以考虑更激进的稀疏化策略。

\textbf{威胁模型限制}:本文主要防御逐字记忆化攻击和成员推断攻击。对于更复杂的推理攻击(如通过模型行为推断用户属性),需要结合其他防御机制。

\textbf{量化误差累积}:在深度网络中,量化误差可能累积。虽然我们的实验表明影响可忽略,但对于更深或更复杂的架构,可能需要逐层校准。

\textbf{侧信道攻击(Side-Channel Attacks)}:尽管物理隔离(分离的内存缓冲区)提供了比算法遗忘更强的确定性,但系统层面仍存在攻击面。由于稀疏流的计算模式与密集流完全不同,攻击者可能通过测量推理时间(Timing Attack)或功耗来推断某个输入是否触发了大量的正交流计算。如果"敏感查询"触发了更多的稀疏计算,这本身就泄露了信息。这是一个重要的系统安全隐患,需要在修订版中深入讨论。未来工作需要探索:
\begin{itemize}
\item \textbf{恒定时间计算}:确保无论$\alpha$值如何,推理时间保持恒定。这可以通过添加虚拟计算或使用固定延迟来实现。
\item \textbf{功耗掩蔽}:通过添加虚拟计算来掩盖真实的稀疏计算模式,使得攻击者无法通过功耗分析推断计算内容。
\item \textbf{查询混淆}:对输入进行预处理,使得敏感查询和普通查询触发相同的计算模式,从而消除时序差异。
\end{itemize}

\textbf{Alpha开关的原子性}:在实际的并发服务中,如何保证$\alpha$参数的切换是原子性的?如果一个请求的一半token用了$\alpha=1$,另一半用了$\alpha=0$,会导致什么后果?这可能导致:
\begin{itemize}
\item \textbf{不一致的推理结果}:同一序列的不同部分使用了不同的模型配置,导致生成文本的逻辑不一致。
\item \textbf{隐私泄露风险}:敏感信息可能在$\alpha$切换的瞬间被泄露,特别是在处理长序列时。
\item \textbf{状态污染}:前一个请求的$\alpha$值可能影响后续请求,导致意外的隐私泄露或性能下降。
\end{itemize}
我们建议在系统实现中采用\textbf{请求级原子性}:每个请求的$\alpha$值在请求开始时确定,并在整个请求处理过程中保持不变。这可以通过以下机制实现:
\begin{itemize}
\item \textbf{请求级别的配置参数}:每个API请求携带$\alpha$参数,在请求处理开始时设置,并在整个序列生成过程中保持不变。
\item \textbf{会话级别的状态管理}:对于需要跨请求保持状态的场景,使用会话ID来管理$\alpha$值,确保同一会话内的所有请求使用相同的$\alpha$值。
\item \textbf{线程本地存储}:在CUDA kernel中,使用线程本地存储来确保每个线程的$\alpha$值在计算过程中不被其他线程修改。
\end{itemize}

\textbf{实例级与种群级曲率的权衡}:虽然我们通过区分实例级和种群级曲率统一了LibOrtho与Merullo的发现,但实际应用中如何平衡这两种视角仍是一个开放问题。未来工作可以探索:
\begin{itemize}
\item \textbf{混合筛选策略}:结合实例级离群值检测和种群级谱分析。
\item \textbf{自适应阈值}:根据数据分布动态调整曲率阈值$\tau$。
\item \textbf{多尺度Hessian分析}:在不同粒度上分析Hessian谱,捕获不同层次的记忆化模式。
\end{itemize}

%-------------------------------------------------------------------------------
\section{相关工作}
%-------------------------------------------------------------------------------

\textbf{量化}:GPTQ~\cite{gptq}、AWQ、SpQR~\cite{spqr}使用类似的数学框架进行模型压缩,但我们的目标是安全而非仅准确性。我们使用他们的数学但反转了目标:不是保留残差以维持准确性,而是将残差作为特权管理以实现隐私隔离。SpQR~\cite{spqr}保留离群值是为了恢复精度,LibOrtho将离群值隔离是为了赋予用户"遗忘的权利"。

\textbf{遗忘}:机器遗忘~\cite{machine-unlearning}试图从已训练模型中移除特定数据,通常需要重新训练或微调,计算成本高昂且结果不确定。我们提供了一种架构替代方案,通过设计实现隔离而非事后删除,在推理时提供确定性的隐私控制。

\textbf{几何解耦遗忘(GU)}:2025年提出的几何解耦遗忘(Geometric-disentanglement Unlearning, GU)~\cite{gu2025}通过动态梯度投影实现遗忘与保留之间的解耦。GU在理论上通过投影到保留集子空间$T_r$的切平面更精准地处理了纠缠问题,但它本质上还是一种"遗忘算法",需要昂贵的多轮梯度下降和投影计算。相比之下,LibOrtho的最大优势在于\textbf{即时性(Instantaneity)}和\textbf{确定性(Determinism)}:LibOrtho不是让模型"学会遗忘",而是让模型"无法回忆"。GU虽然在理论上通过动态投影更精准地处理了纠缠问题,但计算成本为$O(N \cdot d)$($N$为训练样本数,$d$为参数维度),而LibOrtho的预处理成本为$O(d)$,推理时切换成本为$O(1)$。更重要的是,GU提供的是概率性保证(依赖优化结果),而LibOrtho提供的是确定性保证(物理切断数据流)。这种"拔掉U盘"式的物理安全感对法律合规(如GDPR删除权)至关重要。表~\ref{tab:comparison-gu}展示了详细的对比。

\begin{table*}
\centering
\footnotesize
\begin{tabular}{|p{2.5cm}|p{5cm}|p{5cm}|}
\hline
特性 & LibOrtho (本文) & Geometric-disentanglement Unlearning (GU) \\
\hline
\textbf{方法论} & \textbf{架构隔离} (Architectural Isolation) & \textbf{算法优化} (Algorithmic Optimization) \\
\hline
\textbf{实施阶段} & 推理时 (Inference-time) & 再训练/微调时 (Re-training/Fine-tuning) \\
\hline
\textbf{正交性定义} & 静态权重分解 (基于Hessian对角线) & 动态梯度投影 (基于保留集子空间$T_r$) \\
\hline
\textbf{计算成本} & 极低 (一次性离线处理 + 零样本切换) & 高 (需要多轮梯度下降和投影计算) \\
\hline
\textbf{确定性} & \textbf{确定性} (物理切断数据流) & 概率性 (依赖优化结果) \\
\hline
\textbf{主要缺陷} & 对角近似忽略了纠缠;需维护双倍索引 & 难以扩展到超大模型;需访问原始数据 \\
\hline
\end{tabular}
\caption{\label{tab:comparison-gu}LibOrtho与几何解耦遗忘(GU)的对比分析。}
\end{table*}

\textbf{隐私}:差分隐私~\cite{differential-privacy}提供统计保证,但可能显著影响模型性能。我们提供确定性保证:通过设置$\alpha=0$,可以确定性地禁用隐私流,同时保持基础智能能力。

\textbf{损失曲率与记忆化}:Merullo等人~\cite{merullo2025}通过K-FAC分解分析发现,在种群级分析中,通用推理能力依赖于高曲率的共享结构(刚性),而纯粹的记忆化表现为低曲率(平坦)。我们的工作与Merullo的发现是互补的:LibOrtho关注实例级的权重离群值检测,识别为了拟合少数样本而大幅偏离量化格点的"权重异常值"。我们明确区分了实例级曲率与种群级曲率,将两种看似矛盾的理论统一为完整的记忆化几何图景。

\textbf{平坦最小值理论}:我们的理论框架建立在平坦最小值理论~\cite{flat-minima,sharp-minima}之上,该理论表明平坦最小值对应更好的泛化能力。我们扩展了这一理论,证明在实例级分析中,高曲率方向对应记忆化信息。

\textbf{影响函数}:我们使用影响函数~\cite{influence-functions}建立梯度、Hessian和参数更新之间的联系,为我们的记忆化定义提供理论基础。然而,我们承认将影响函数应用于LLM时,逆Hessian向量积(iHVP)的估计存在误差~\cite{influence-llm-2024},因此我们的理论推导是启发式的而非严格的。

\textbf{遗忘评估基准}:2024-2025年涌现了多个遗忘评估基准,包括TOFU~\cite{tofu2024}(虚构遗忘任务)、WMDP~\cite{wmdp2024}(危险知识消除)和MUSE~\cite{muse2024}(六维遗忘评估)。这些基准不仅考察逐字记忆,还深入考察概率性泄漏(如成员推断攻击)和模型效用的保留情况。我们在本文中全面采用了这些最新基准,证明了LibOrtho在统计意义上的有效性。

%-------------------------------------------------------------------------------
\section{结论}
%-------------------------------------------------------------------------------

我们证明了隐私不是数据的属性,而是\textbf{模型参数几何}的属性。LibOrtho通过尊重系统设计中的这种几何结构,可以在不牺牲通用智能的情况下实现可信AI。我们的工作为LLM安全开辟了新的研究方向,将几何理论与系统实现相结合。

\textbf{架构隐私:AI合规的基础设施层}。LibOrtho的核心贡献不仅在于技术实现,更在于提出了"架构隐私"(Architectural Privacy)这一全新的概念框架。与传统的算法级隐私保护(如差分隐私、机器遗忘)不同,架构隐私通过在系统设计层面实现物理隔离,提供了确定性的隐私控制机制。这种"拔掉U盘"式的物理安全感对法律合规(如GDPR删除权、CCPA数据删除要求)至关重要,因为它不依赖于概率性保证,而是通过架构设计确保隐私数据的物理隔离。我们相信,架构隐私将成为未来可信AI系统的基础设施层,为AI合规提供坚实的技术基础。

\textbf{核心贡献总结}:
\begin{itemize}
\item \textbf{理论统一}:我们通过区分实例级与种群级曲率,统一了LibOrtho与Merullo等人看似矛盾但实则互补的发现,揭示了记忆化的完整几何图景。
\item \textbf{架构创新}:LibOrtho提出了"架构隔离"而非"算法后处理"的范式,通过物理隔离实现确定性的隐私控制,而非概率性保证。
\item \textbf{评估现代化}:我们全面采用了2025年的最新评估基准(TOFU、WMDP、MUSE),证明了LibOrtho不仅在逐字输出层面,更在概率分布和统计推断层面消除了隐私痕迹。
\item \textbf{概念创新}:我们提出了"分辨率缩放智能"(Resolution-Scaled Intelligence)和"脆性知识与鲁棒知识的几何分离"等新概念,将数学能力下降从"缺陷"重构为"核心特性"。更重要的是,我们揭示了精确推理(如数学计算)与机械记忆在几何结构上的相似性,证明了LibOrtho成功分离了"脆性知识"(包括隐私和精确计算)与"鲁棒知识"(模糊推理和语法),这不仅解决了隐私问题,更揭示了智能的几何本质。
\item \textbf{架构隐私范式}:我们提出了"架构隔离"而非"算法后处理"的范式,将隐私保护从算法层面提升到架构层面,为AI合规提供了基础设施级的技术支撑。
\end{itemize}

\textbf{未来方向}:LibOrtho开启了"可信推理架构"这一全新的研究方向。未来工作可以探索多级Ortho架构、更高效的Hessian近似方法、以及对抗侧信道攻击的防御机制。我们相信,通过将几何理论与系统实现相结合,可以构建更加安全、可控、可信的大语言模型。

%-------------------------------------------------------------------------------
\section*{致谢}
%-------------------------------------------------------------------------------

\textbf{注意:提交时请勿包含可能使您去匿名化的致谢(例如,因特定隶属关系或资助而致谢)}

%-------------------------------------------------------------------------------
% optional clearing of the page
\cleardoublepage
\appendix
\section*{伦理考量}
\textbf{在一页以内,解释您工作的伦理考量。此附录必须使用此确切标题,否则可能面临桌面拒绝。在提交论文前请仔细研究伦理指南。}

本研究涉及大型语言模型的隐私和安全问题。我们开发的技术旨在帮助用户控制模型中的隐私信息。然而,我们也认识到:

\begin{itemize}
\item \textbf{双重用途(Dual Use)}:我们的技术可能被恶意行为者用于隐藏模型中的敏感训练数据,也可能被用于保护用户隐私。我们强调负责任的使用。

\textbf{恶意使用场景}:一个潜在的滥用场景是,攻击者可能只分发Base模型,将恶意代码、后门或有害内容隐藏在Ortho流中,作为"激活码"分发。当用户启用Ortho流($\alpha=1$)时,恶意内容被激活。虽然LibOrtho本身无法完全解决这个问题(这是模型分发和信任的问题),但我们认识到这种风险,并建议:
\begin{enumerate}
\item \textbf{模型审计}:在部署前,对Ortho流进行内容审计,检测潜在的恶意模式。
\item \textbf{访问控制}:在可信环境中,对Ortho流的访问进行严格的身份验证和授权。
\item \textbf{透明度机制}:提供工具让用户检查Ortho流的内容,提高模型的可解释性。
\end{enumerate}

虽然我们无法完全消除这种风险,但讨论它显示了我们对技术潜在滥用的深度认识,这也是负责任研究的一部分。

\item \textbf{模型透明度}:通过允许用户禁用模型的某些部分,我们可能降低模型的可解释性。需要在隐私和透明度之间取得平衡。我们建议在隐私保护场景下,提供Base模型的完整可解释性,而Ortho流的内容可以在可信环境中进行审计。

\item \textbf{公平性}:我们的方法可能对不同类型的数据产生不同的影响。需要进一步研究以确保公平性。特别是,如果某些群体的隐私信息更容易被编码到Ortho流中,可能导致不公平的隐私保护水平。
\end{itemize}

所有实验均在受控环境中进行,使用的数据集已获得适当许可。我们遵循了相关机构的伦理审查程序。

% optional clearing of the page
\cleardoublepage

\section*{开放科学}
\textbf{在一页以内,此附录必须列出评估论文贡献所需的所有工件,并明确说明审查委员会如何访问每个工件。此附录必须使用此确切标题,否则可能面临桌面拒绝。}

为了促进可重现性和开放科学,我们提供以下工件:

\begin{itemize}
\item \textbf{源代码}:LibOrtho的完整源代码可在匿名GitHub仓库获得:\texttt{https://anonymous.4open.science/r/libortho}。代码包括:
  \begin{itemize}
  \item Hessian筛选器的实现
  \item 融合双GEMM内核(CUDA)
  \item 评估脚本和实验配置
  \end{itemize}

\item \textbf{数据集}:
  \begin{itemize}
  \item 合成Canary数据集:包含在代码仓库中
  \item 使用的公开数据集(WikiText-2、C4、MMLU):标准基准,可从原始来源获取
  \end{itemize}

\item \textbf{模型检查点}:
  \begin{itemize}
  \item 预处理的Llama-2-7B和Llama-3-8B模型(基础流+正交流)可通过匿名链接获取
  \item 由于存储限制,仅提供处理后的模型权重,不包含原始训练数据
  \end{itemize}

\item \textbf{实验脚本}:
  \begin{itemize}
  \item 所有评估脚本包含在代码仓库的\texttt{eval/}目录中
  \item 包含详细的README说明如何复现所有实验结果
  \end{itemize}

\item \textbf{访问方式}:
  \begin{itemize}
  \item 代码和脚本:GitHub(匿名)
  \item 模型检查点:匿名云存储链接(在README中提供)
  \item 所有链接在论文接受后将更新为永久链接
  \end{itemize}
\end{itemize}

% optional clearing of the page
\cleardoublepage
\bibliographystyle{plain}
\bibliography{\jobname}

%%%%%%%%%%%%%%%%%%%%%%%%%%%%%%%%%%%%%%%%%%%%%%%%%%%%%%%%%%%%%%%%%%%%%%%%%%%%%%%%
\end{document}
%%%%%%%%%%%%%%%%%%%%%%%%%%%%%%%%%%%%%%%%%%%%%%%%%%%%%%%%%%%%%%%%%%%%%%%%%%%%%%%%

